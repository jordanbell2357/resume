\documentclass[11pt,a4paper,sans]{moderncv}
\moderncvstyle{banking} % style options are 'casual' (default), 'classic', 'oldstyle', and 'banking'
\moderncvcolor{blue} % color options 'blue' (default), 'orange', 'green', 'red', 'purple', 'grey' and 'black'
\usepackage[scale=0.87]{geometry} % adjust the page margins
\usepackage{microtype} % Better typography
\usepackage{fontawesome5}



% Personal data
\name{Jordan}{Bell}
\phone[mobile]{416-528-3258}
\email{jordan.bell@gmail.com}
\social[linkedin]{jordanbell2357}
\social[github]{jordanbell2357}
\social[kaggle]{jordanbell2357}
\homepage{jordanbell.info}
\extrainfo{\href{https://www.credly.com/users/jordanbell2357/badges}{Credly Profile}}

\begin{document}
\makecvtitle

% Education
\section{Education}
\cventry{2018--2019}{Graduate Certificate}{Analytics for Business Decision Making}{George Brown College}{Toronto}{}
\cventry{2007--2009}{Master of Science}{Department of Mathematics}{University of Toronto}{Toronto}{Canada Graduate Scholarships, Master’s (CGS M)}
\cventry{2003--2007}{Bachelor of Mathematics}{Mathematics}{Carleton University}{Ottawa}{University Medal in Mathematics}


\section{Publications}

\cvitem{}{
Andrews, George E., and Jordan Bell. “Euler’s Pentagonal Number Theorem and the Rogers-Fine Identity.” Annals of Combinatorics 16, no. 3 (2012): 411–20. \url{https://doi.org/10.1007/s00026-012-0139-4}. \href{https://zbmath.org/?q=an\%3A1256.05018}{Zbl 1256.05018}
}

\cvitem{}{
Bell, Jordan. “A New Method for Constructing Nonlinear Modular n-Queens Solutions.” Ars Combinatoria 78 (2006): 151–55. \href{https://zbmath.org/?q=an\%3A1164.05327}{Zbl 1164.05327}
}

\cvitem{}{
Bell, Jordan. “A Summary of Euler’s Work on the Pentagonal Number Theorem.” Archive for History of Exact Sciences 64, no. 3 (2010): 301–73. \url{https://doi.org/10.1007/s00407-010-0057-y}. \href{https://zbmath.org/?q=an\%3A1208.01013}{Zbl 1208.01013}
}

\cvitem{}{
Bell, Jordan. “Cyclotomic Orthomorphisms of Finite Fields.” Discrete Applied Mathematics 161, no. 1–2 (2013): 294–300. \url{https://doi.org/10.1016/j.dam.2012.08.013}. \href{https://zbmath.org/?q=an\%3A1364.11155}{Zbl 1364.11155}
}

\cvitem{}{
Bell, Jordan. “Estimates for the Norms of Products of Sines and Cosines.” Journal of Mathematical Analysis and Applications 405, no. 2 (2013): 530–45. \url{https://doi.org/10.1016/j.jmaa.2013.04.010}. \href{https://zbmath.org/?q=an\%3A1311.11098}{Zbl 1311.11098}
}

\cvitem{}{
Bell, Jordan. “Nonlinear Modular Latin Queen Squares.” Utilitas Mathematica 74 (2007): 71–75. \href{https://zbmath.org/?q=an\%3A1170.05015}{Zbl 1170.05015}.
}

\cvitem{}{
Bell, Jordan. “Polynomial Modular n-Queens Solutions.” Acta Arithmetica 129, no. 4 (2007): 335–39. \url{https://doi.org/10.4064/aa129-4-4}. \href{https://zbmath.org/?q=an\%3A1140.11057}{Zbl 1140.11057}
}

\cvitem{}{
Bell, Jordan, and Viktor Blåsjö. “Pietro Mengoli’s 1650 Proof that the Harmonic Series Diverges.” Mathematics Magazine 91, no. 5 (2018): 341–47. \url{https://doi.org/10.1080/0025570X.2018.1506656}. \href{Zbl 1407.40001}{https://zbmath.org/?q=an\%3A1407.40001}. \href{https://www.maa.org/programs-and-communities/member-communities/maa-awards/writing-awards/carl-b-allendoerfer-awards}{2019 recipient of Carl B. Allendoerfer Award for expository mathematical writing, Mathematical Association of America (MAA)}
}

\cvitem{}{
Bell, Jordan, and Brett Stevens. “A Survey of Known Results and Research Areas for n-Queens.” Discrete Mathematics 309, no. 1 (2009): 1–31. \url{https://doi.org/10.1016/j.disc.2007.12.043}. \href{https://zbmath.org/?q=an\%3A1228.05002}{Zbl 1228.05002}
}

\cvitem{}{
Bell, Jordan, and Brett Stevens. “Constructing Orthogonal Pandiagonal Latin Squares and Panmagic Squares from Modular n-Queens Solutions.” Journal of Combinatorial Designs 15, no. 3 (2007): 221–34. \url{https://doi.org/10.1002/jcd.20143}. \href{https://zbmath.org/?q=an\%3A1117.05016}{Zbl 1117.05016}. Cited by 250+ publications.
}

\cvitem{}{
Bell, Jordan, and Brett Stevens. “Results for the n-Queens Problem on the Möbius Board.” The Australasian Journal of Combinatorics 42 (2008): 21–34. \href{https://zbmath.org/?q=an\%3A1175.05027}{Zbl 1175.05027}
}

\cvitem{}{
Bell, Jordan, and Qiang Wang. “Results on Permutations with Distinct Difference Property.” Contributions to Discrete Mathematics 4, no. 1 (2009): 107–11. \href{https://zbmath.org/?q=an\%3A1203.05002}{Zbl 1203.05002}
}


\section{Professional Experience}
\cventry{November 2023--February 2024}{Business Intelligence Developer}{Donor Compass}{Toronto, Ontario, Canada (Remote)}{}{
\begin{itemize}%
\item Facilitated secure data transfers with clients, focusing on optimizing data fields and filters for enhanced processing efficiency.
\item Pioneered the transformation of raw data into actionable insights using advanced data manipulation and engineering techniques.
\item Innovated and maintained PowerBI dashboards, integrating client data with scoring algorithms to illustrate donor propensity.
\item Led the modernization of data sharing infrastructure from traditional VPN and Samba sharing to Citrix ShareFile, significantly improving security and accessibility.
\item Transitioned the legacy scoring engine to a virtual machine environment, managing configurations for PHP, Redis, MySQL, and Liquibase.
\item Overhauled client data management systems, establishing systematic databases and assuming the role of database administrator.
\item Employed Python and pandas for efficient data cleaning, complementing and enhancing SQL-based data processes.
\item Utilized PostgreSQL for robust database management, handling large-scale datasets with a focus on accuracy and speed.
\item Proactively addressed and resolved VPN and network-related issues, ensuring smooth and uninterrupted operations.
\item Left a strong impact in a short tenure before the company's insolvency; received commendation from manager, available for reference.
\end{itemize}}


\cventry{June 2022--August 2023}{Data Science Associate}{Canadian Tire}{Toronto}{}{
\begin{itemize}%
\item Developed store similarity metrics comparing store sales at any level of aggregation of product.
\item Participated in planning and creating pipeline for Google Analytics page view data into OLAP database, for store similarity calculations
\item Orchestrated the integration of geospatial census data and Environics data, linking regional characteristics to surrounding Canadian Tire stores by postal code to inform strategic business decisions.
\item Developed a robust and dynamic OLAP database table view to produce a dashboard for monitoring dealer participation in promotional deals. This involved complex windowing functions and time span standardization using step functions for variable deal lengths, facilitating real-time and historical analysis.
\end{itemize}}

\cventry{January 2019--April 2019}{Data Science Intern}{Consilium Crypto}{Toronto}{}{
\begin{itemize}%
\item Engineered a novel approach to feature engineering for time series data, integrating price and volume data from multiple cryptocurrency exchanges with blockchain transaction data to enhance predictive model robustness.
\item Designed and tested a logistic regression model to analyze Ethereum data, identifying key indicators that influence price and volume movements.
\item Utilized domain knowledge to select significant time periods revered by the trading community, which informed the feature engineering process and enriched the model's predictive power.
\end{itemize}}

\cventry{January 2021--June 2022}{Mathematics Tutor}{Jordan Bell Tutoring}{Toronto}{}{}

\cventry{March 2018--January 2021}{Mathematics Tutor}{Toronto Elite Tutorial Services}{Toronto}{}{}

\cventry{April 2013--April 2017}{Mathematics Course Instructor}{University of Toronto}{Toronto}{}{
\begin{itemize}%
\item Mentored students and developed course materials.
\item Organized teaching assistant duties, time allocation, and preparation of materials for use in tutorials.
\item Instructor for multivariable calculus, ordinary differential equations, and linear algebra courses, as single instructor and as part of multiple section courses (first and second year courses with versions for different programs).
\end{itemize}}

\cventry{September 2009 - April 2013}{Mathematics Teaching Assistant}{University of Toronto}{Toronto}{}{
\begin{itemize}%
\item 1st year courses: Calculus for engineers, computer science, biology, and math specialists (separate courses), and linear algebra for engineers and computer science. 2nd year courses: Mathematical writing (essay marking), linear programming, ordinary differential equations for computer science and for math specialists (separate courses). 3rd year courses: Complex analysis, functional analysis, group theory, partial differential equations, dynamical systems. 4th year courses: Nonlinear optimization
\end{itemize}}


\end{document}